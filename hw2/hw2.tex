\documentclass[12pt]{article}
%\usepackage[framed,numbered,autolinebreaks,useliterate]{mcode}
\usepackage[margin=1.0in]{geometry}
\title{HW \#2}
\author{
        Travis Collins \\
	traviscollins@wpi.edu \\
	ECE 578 Crytography and Data Security
}
\date{\today}
\usepackage{titling}
\setlength{\droptitle}{-1.0in}


\begin{document}
\maketitle

\section{Question 1: Linear Complexity}
Note: All answers used Matlab code appended to end of homework.\\
\textbf{a 0101010101}
\[ X = \left( \begin{array}{ccc}
0 & 1 & 0 \\
1 & 0 & 1 \\
0 & 1 & 0 \end{array} \right)  rank(X) = 2\] 
Linear Complexity: m=2\\

\textbf{b 011001100110}
\[ X = \left( \begin{array}{cccc}
0 & 1 & 1 & 0 \\
0 & 0 & 1 & 1 \\
1 & 0 & 0 & 1 \\
1 & 1 & 0 & 0 \end{array} \right)  rank(X) = 3\] 
Linear Complexity: m=3\\

\textbf{c 011011011011011}
\[ X = \left( \begin{array}{cccc}
0 & 1 & 1 & 0  \\
1 & 0 & 1 & 1  \\
1 & 1 & 0 & 1  \\
0 & 1 & 1 & 0\end{array} \right)  rank(X) = 3\] 
Linear Complexity: m=3\\

\textbf{d 1011010010110}
\[ X = \left( \begin{array}{cccccc}
1 & 0 & 1 & 1 & 0 & 1\\
0 & 1 & 0 & 1 & 1 & 0\\
0 & 0 & 1 & 0 & 1 & 1\\
1 & 0 & 0 & 1 & 0 & 1\\
0 & 1 & 0 & 0 & 1 & 0\\
1 & 0 & 1 & 0 & 0 & 1\end{array} \right)  rank(X) = 5\] 
Linear Complexity: m=5\\


\section{Question 2}
\textbf{A: Degree of stream generator}
\[m=3\]\\

\textbf{B: Initialization Vector}
\[ S=[0, 0, 1] \]

\textbf{C: Feedback Coefficients}
\[ F=\left( \begin{array}{c}
0\\
1\\
1\end{array} \right) \]

\section{Question 3}
No this does not significantly improve over using single a LFSR.  By XORing two LFSR's together we can at most double their complexity.  For example if we take two LFSR of length 3, with coefficients \[ F1=\left( \begin{array}{c}
1\\
0\\
1\end{array} \right) 
 F2=\left( \begin{array}{c}
1\\
1\\
1\end{array} \right)\]\\

If they are XORed together, the linear complexity of their output at most is m=6, depending on input.

\section{Question 4}
\textbf{b: Output}\\
\[Output=[0 1 0 0 1 1 1 1] \]

\textbf{c: Sequence length}\\
Yes the condition of the length's being relatively prime is true, and the length of the output sequence is: 

\begin{table}
    \begin{tabular}{|l|l|l|l|l|l|l|l|l|l|}
        $Clock$ & $Initial$ & 1 & 2 & 3 & 4 & 5 & 6 & 7 & 8 \\
        $LFSR1$ &       1 & 0 & 1 & 1 & 0 & 1 & 1 & 0 & 1 \\ 
        $LFSR2$ &	1 & 0 & 0 & 0 & 0 & 1 & 1 & 1 & 1  \\
        $LFSR3$ &	1 & 1 & 0 & 0 & 0 & 0 & 0 & 0 & 0  \\ 
	$Output$ &      0 & 1 & 0 & 0 & 0 & 1 & 1 & 1 & 1 \\
    \end{tabular}
\end{table}

 
\end{document}
