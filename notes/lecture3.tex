\documentclass[12pt,letterpaper]{article}

\usepackage{amsmath} % just math
\usepackage{amssymb} % allow blackboard bold (aka N,R,Q sets)
\usepackage{ulem}
\linespread{1.6}  % double spaces lines
\usepackage[left=1in,top=1in,right=1in,bottom=1in,nohead]{geometry}
\usepackage{enumerate}

\begin{document}

\linespread{1} % single spaces lines
\small \normalsize %% dumb, but have to do this for the prev to work
\begin{flushright}
Travis Collins \\
ECE578 Lecture 3 Notes \\
\today
\end{flushright}

\underline{RC4  \(Ron's  Cipher 4\)}\\
-Designed in 1987\\
-Trade Secret of RSA Corp\\
-Leaked on sci.crypt \(USENET\) in 1994\\
-Most widely used stream cipher, SSL/TLS, WEP/WPA\\
-Key Advantage: Amazingly simple/easy to implement!!\\
-RC4 works with bytes (8-bits) and not bits\\
-RC4 State
\begin{enumerate}\itemsep0.5pt
\item A 256 byte state table
\item Two 8-bit indices i, j
\end{enumerate}


\underline{RC4 Key Schedule Algorthim}\\
-Prepares the state table S using a short key or password\\
-Key has to be at least 1 byte \[1<=\left|{key}\right|<=256 bytes,key=n\]\\


\underline{Algorithm}\\
for i=0 to 255: S[i]=i\\
j=0\\
for i=0 to 255:\\
	j=(j+S[i]+key[i mod n])mode 256\\
	swap(S[i],S[j])\\

\underline{Key Stream Generation Algorithm}\\
-In each iteration we generate a byte of keystream data\\
-Initially set i=j=0 (Only at beginning of encryption session)\\

\underline{Algorithm}\\
i=i+1 (mod 256)\\
j=j+S[i] (mod 256) \\
swap(S[i],S[j])\\
return S[ S[i]+S[j] (mod 256)] \\


\underline{Block Ciphers}
-Remember (from stream ciphers) PRNG output string is indistinguishable from a random string for any bounded adversary\\

\textbf{Idea:} What if we can randomize the function itself instead of the output of the function!\\
-Pseudo Random Function (PRF)\\
\textbf{Def:} A PRF is a keyed function that is indistinguishable from a function chosen at random using bounded resources\\

\underline{Block Cipher}(Approx. of Pseudo-Random Permutation (PRP))\\
 is stateless meaning the same message and key in means the same cipher text out \[E_{k}(m)=c D_{k}(E_{k}(m))=m)\]


\underline{An Ideal Block Cipher}
-Assume we fix \[n_{k}=n_{m}=n_{c}=n\]\\
-What we need is a random function n-bit to n-bit function\\
\vspace{0.1 mm}
-Consider first all functions\\

\begin{tabular}{ l | r }
  \hline                        
  Message & Cipher \\
  0 & $2^n$ \\
  1 & $2^n$ \\
  ... & ... \\
  $2^n-1$ & $2^n$ \\
  \hline  
\end{tabular}

\[\left|{F}\right|=(2^n)^{2^n} \]\\
\vspace{0.1 mm}
-But we want decryption to work so f needs to be one-to-one\\

\begin{tabular}{ l | r }
  \hline                        
  Message & Cipher \\
  0 & $2^n$ \\
  1 & $2^n-1$ \\
  ... & ... \\
  $2^n-1$ & 1 \\
  \hline  
\end{tabular}

\[\left|{F}\right|=2^n(2^n-1)(2^n-2)...=(2^n)! \] 
Still huge space!!\\
But we also want it to be efficiently computable\\

Can we construct a PRP from a PRF?\\
- Luby-Rackoff Construction-(Feistel Cipher)\\
- Look at DES paper\\


\end{document}
